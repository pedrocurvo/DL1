\documentclass{article}

% if you need to pass options to natbib, use, e.g.:
%     \PassOptionsToPackage{numbers, compress}{natbib}
% before loading neurips_2024


% ready for submission
% \usepackage{neurips_2024}


% to compile a preprint version, e.g., for submission to arXiv, add add the
% [preprint] option:
     \usepackage[preprint]{neurips_2024}


% to compile a camera-ready version, add the [final] option, e.g.:
%     \usepackage[final]{neurips_2024}


% to avoid loading the natbib package, add option nonatbib:
%    \usepackage[nonatbib]{neurips_2024}


\usepackage[utf8]{inputenc} % allow utf-8 input
\usepackage[T1]{fontenc}    % use 8-bit T1 fonts
\usepackage{hyperref}       % hyperlinks
\usepackage{url}            % simple URL typesetting
\usepackage{booktabs}       % professional-quality tables
\usepackage{amsfonts}       % blackboard math symbols
\usepackage{nicefrac}       % compact symbols for 1/2, etc.
\usepackage{microtype}      % microtypography
\usepackage{xcolor}         % colors
\usepackage{amsmath}
\usepackage{natbib}
\usepackage{graphicx}  % For including graphics
\usepackage{float}     % To control figure placement
\usepackage{geometry}
\usepackage{array}
\usepackage{multirow}
\usepackage{graphicx}
\usepackage{subcaption}



\title{Deep Learning 1 - Homework 3}


% The \author macro works with any number of authors. There are two commands
% used to separate the names and addresses of multiple authors: \And and \AND.
%
% Using \And between authors leaves it to LaTeX to determine where to break the
% lines. Using \AND forces a line break at that point. So, if LaTeX puts 3 of 4
% authors names on the first line, and the last on the second line, try using
% \AND instead of \And before the third author name.


\author{%
  Pedro M.P.~Curvo \\
  MSc Artificial Intelligence\\
  University of Amsterdam\\
  \texttt{pedro.pombeiro.curvo@student.uva.nl} \\
  % examples of more authors
  % \And
  % Coauthor \\
  % Affiliation \\
  % Address \\
  % \texttt{email} \\
  % \AND
  % Coauthor \\
  % Affiliation \\
  % Address \\
  % \texttt{email} \\
  % \And
  % Coauthor \\
  % Affiliation \\
  % Address \\
  % \texttt{email} \\
  % \And
  % Coauthor \\
  % Affiliation \\
  % Address \\
  % \texttt{email} \\
}


\begin{document}


\maketitle


% \begin{abstract}
%   The abstract paragraph should be indented \nicefrac{1}{2}~inch (3~picas) on
%   both the left- and right-hand margins. Use 10~point type, with a vertical
%   spacing (leading) of 11~points.  The word \textbf{Abstract} must be centered,
%   bold, and in point size 12. Two line spaces precede the abstract. The abstract
%   must be limited to one paragraph.
% \end{abstract}


\section*{Part 1}

\subsection*{1.1}

To sample an image using the decoder $f_{\theta}$ from the generative model described in this section, we first
need to sample a latent vector $z$ from the prior distribution $p(z)$. In this case, the prior distribution is a
standard multivariate Gaussian distribution, so we can sample $z$ from a standard normal distribution, i.e., $z \sim \mathcal{N}(0, I_D)$.
Then, we need to compute the pixel probabilities using the decoder $f_{\theta}$. For this, we pass
the sampled latent vector $z$ through the decoder $f_{\theta}$, which is a neural network parameterized by $\theta$.
This will map the latent vector $z$ to the probabilities of the Categorical distribution for each pixel $x_m$ in the image.
$f_{\theta}(z) \rightarrow (p_1, p_2, ..., p_k)^M$. 
Here: $p_m = (p_{m1}, p_{m2}, ..., p_{mk})$ are the event probabilities for pixel $m$ being in one of the $k$ categories.
$M$ is the number of pixels in the image. 
The, we can sample pixel values $x_n$ for each pixel $m$ by sampling from the Categorical distribution $Cat(x_m | f_{\theta}(z)_m)$.
Where, $x_m \sim Cat(p_m)$. This will generate the image by sampling each pixel value independently based on the
probabilities computed by the decoder. 
Finally we combine the pixel values $x_m$ to from the image $x_n$. 

\subsection*{1.2}

Monte Carlo integration with samples from $p(z_n)$ can be used to approximate the expectation of the log-likelihood. 
However, is inefficient for training VAE models because it requires a large number of samples to accurately estimate the
log-likelihood. This problem appears because the dimensionality of the latent space $z$ affects the number of samples
needed to cover the entire latent space adequately. As the dimensionality of the latent space increases, the space
becomes sparser, and more samples are needed to cover the space to capture the posterior distribution $p(z|x)$, which
is often highly concentrated in certain regions of the latent space (as shown by the blue contours in Figure 2).

This results in a exponentially increasing number of samples needed to accurately estimate the log-likelihood, 
that is, $L \rightarrow \infty$ as $D \rightarrow \infty$. This makes Monte Carlo integration impractical for high-dimensional
latent spaces due to the computational cost of sampling a large number of points to estimate the log-likelihood accurately.
Therefore, other methods are used to estimate the log-likelihood in VAE models. 

\subsection*{1.3}

From Equation 10, we have:

\begin{align*}
    \log p(x_n) - KL(q_{\theta}(z_n|x_n) || p(z_n|x_n)) &= \mathbb{E}_{q_{\theta}(z_n|x_n)}[\log p(x_n|z_n)] - KL(q_{\theta}(z_n|x_n) || p(z_n)) \\
\end{align*}

Rearranging the equation we get: 

\begin{align*}
    \log p(x_n) &= \mathbb{E}_{q_{\theta}(z_n|x_n)}[\log p(x_n|z_n)] - KL(q_{\theta}(z_n|x_n) || p(z_n)) + KL(q_{\theta}(z_n|x_n) || p(z_n|x_n)) \\
\end{align*}

Now, we know the KL divergence is always non-negative, since it is a measure of the difference between two distributions by
measuring how much one probability distribution diverges from a second, expected probability distribution. It is zero if and only if the two distributions are the same.
Mathematically, we have that: 

\begin{align*}
    KL(q_{\theta}(z_n|x_n) || p(z_n)) &\geq 0 \\
    KL(q_{\theta}(z_n|x_n) || p(z_n|x_n)) &\geq 0 \\
\end{align*}

With this we have that: 

\begin{align*}
  \log p(x_n) &= \mathbb{E}_{q_{\theta}(z_n|x_n)}[\log p(x_n|z_n)] - KL(q_{\theta}(z_n|x_n) || p(z_n)) + KL(q_{\theta}(z_n|x_n) || p(z_n|x_n)) \\
  &\geq \mathbb{E}_{q_{\theta}(z_n|x_n)}[\log p(x_n|z_n)] - KL(q_{\theta}(z_n|x_n) || p(z_n)) \\
\end{align*}

Therefore, the right-hand side of the equation $\mathbb{E}_{q_{\theta}(z_n|x_n)}[\log p(x_n|z_n)] - KL(q_{\theta}(z_n|x_n) || p(z_n))$ is
always less than or equal to the true log-likelihood $\log p(x_n)$. Hence, the right-hand side of the equation is a lower bound on the true log-likelihood.


\subsection*{1.4}

ELBO consists of two terms: 

\begin{align*}
    ELBO(x_n) &= \mathbb{E}_{q_{\theta}(z_n|x_n)}[\log p(x_n|z_n)] - KL(q_{\theta}(z_n|x_n) || p(z_n)) \\
\end{align*}

The first term, $\mathbb{E}_{q_{\theta}(z_n|x_n)}[\log p(x_n|z_n)]$, represents the expected log-likelihood of the data $x_n$ given the latent variable $z_n$.
However, this value stays the same regardless of how $q(z_n|x_n)$ is chosen.
The second term, $KL(q_{\theta}(z_n|x_n) || p(z_n))$, represents the KL divergence between the approximate posterior $q_{\theta}(z_n|x_n)$ and the prior $p(z_n)$.
As $q_{\theta}(z_n|x_n)$ approaches the true posterior $p(z_n|x_n)$, the KL divergence term decreases towards zero. This
is because the KL divergence is minimized when the two distributions are the same, $q(z_n|x_n) = p(z_n|x_n)$.
With that being said, as the KL divergence term decreases, the ELBO increases. becoming closer to the true log-likelihood $\log p(x_n)$.
Ideally, when $q(z_n|x_n) = p(z_n|x_n)$, the ELBO is equal to the true log-likelihood $\log p(x_n)$.

So: 

\begin{align*}
    \lim_{q_{\theta}(z_n|x_n) \rightarrow p(z_n|x_n)} ELBO(x_n) &= \lim_{q_{\theta}(z_n|x_n) \rightarrow p(z_n|x_n)} \mathbb{E}_{q_{\theta}(z_n|x_n)}[\log p(x_n|z_n)] - KL(q_{\theta}(z_n|x_n) || p(z_n)) \\
    &= \mathbb{E}_{p(z_n|x_n)}[\log p(x_n|z_n)] - \lim_{q_{\theta}(z_n|x_n) \rightarrow p(z_n|x_n)} KL(q_{\theta}(z_n|x_n) || p(z_n)) \\
    &= \mathbb{E}_{p(z_n|x_n)}[\log p(x_n|z_n)] - KL(p(z_n|x_n) || p(z_n)) \\
    &= \mathbb{E}_{p(z_n|x_n)}[\log p(x_n|z_n)] - 0 \\
    &= \mathbb{E}_{p(z_n|x_n)}[\log p(x_n|z_n)] \\
    &= \log p(x_n) \\
\end{align*}

And that is why the main goal of training a VAE is to maximize the ELBO, as it is a lower bound on the true log-likelihood $\log p(x_n)$.
And this corresponds to minimizing the KL divergence between the approximate posterior $q_{\theta}(z_n|x_n)$ and the prior $p(z_n)$.

\subsection*{1.5}

The names \textbf{reconstruction} loss and \textbf{regularization} loss are used because: 

\begin{itemize}
  \item \textbf{Reconstruction Loss}: The term $\mathbb{E}_{q_{\theta}(z_n|x_n)}[\log p(x_n|z_n)]$ is the expected log-likelihood
  of the data $x_n$ given the latent variable $z_n$. Hence, it is a measure of how well the model can reconstruct the
  observed data $x_n$ from the latent variable $z_n$. Meaning, it directly corresponds to the task of reconstructing the
  original input, being the main objective of the model. Therefore, it is called the \textbf{reconstruction} loss.

  \item\textbf{Regularization Loss:} The term $KL(q_{\theta}(z_n|x_n) || p(z_n))$ is the KL divergence and ensures
that the learned posterior distribution $q_{\theta}(z_n|x_n)$ is close to the prior distribution $p(z_n)$. This term
regularizes the model by preventing the posterior distribution from deviating too much from the prior distribution.
Therefore, it is called the \textbf{regularization} loss.
\end{itemize}

\subsection*{1.6}

When the prior distribution $p(z_n)$ is and variational posterior $q_{\theta}(z_n|x_n)$ are not Gaussian, the KL divergence
term $KL(q_{\theta}(z_n|x_n) || p(z_n))$ cannot be computed simply in closed form. To overcome this, we can use the
Monte Carlo approximation to estimate the regularization term. 

The regularization term in the VAE is the KL divergence between the variational distribution $q_{\theta}(z_n|x_n)$ and the prior distribution $p(z_n)$.
Mathematically, this term is defined as:

\begin{align*}
    D_{KL}(q_{\theta}(z_n|x_n) || p(z_n)) &= \mathbb{E}_{q_{\theta}(z_n|x_n)}[\log \frac{q_{\theta}(z_n|x_n)}{p(z_n)}] \\
\end{align*}

Assuming the closed-form expression for the KL divergence is intractable, we can use the Monte Carlo approximation to estimate the KL divergence.
First, we sample $L$ latent vectors $z_n^{(l)}$ from the variational distribution $q_{\theta}(z_n|x_n)$, where $l = 1, 2, ..., L$.
Then, we compute the log-ratio of the variational distribution and the prior distribution for each sample:

\begin{align*}
    \log \frac{q_{\theta}(z_n^{(l)}|x_n)}{p(z_n)} \\
\end{align*}

where $q_{\theta}(z_n^{(l)}|x_n)$ is the probability density of the variational distribution evaluated at the sample $z_n^{(l)}$.
Finally, we compute the Monte Carlo estimate of the KL divergence as:

\begin{align*}
    D_{KL}(q_{\theta}(z_n|x_n) || p(z_n)) &\approx \frac{1}{L} \sum_{l=1}^{L} \log \frac{q_{\theta}(z_n^{(l)}|x_n)}{p(z_n)} \\
\end{align*}

Then, we can use this approximation in the loss function:

\begin{align*}
    \mathbb{L}_{regularization,n} &\approx \frac{1}{L} \sum_{l=1}^{L} \log \frac{q_{\theta}(z_n^{(l)}|x_n)}{p(z_n)} \\
\end{align*}

This methods allows us to estimate the KL divergence term when the prior and variational posterior are not Gaussian. The more
samples we use, the more accurate the estimate will be, but it will also increase the computational cost of training the model.
This method can be used for any arbitrary prior and variational posterior distributions, as long as we can sample from the variational distribution
and compute the log-ratio of the two distributions.

It is also important to note the in this case the sampling is more efficient than the Monte Carlo integration for estimating the log-likelihood.
Sampling from $p(z_n)$ can lead to a more inefficient sampling because the prior may not capture the actual structure of the data, 
that is, many samples may be drawn from regions of $z_n$ that do not contribute to $p(x_n | z_n)$, leading to a poor estimate of the log-likelihood,
as seen in the image, where the prior distribution is a standard normal distribution. However, $p(z_n | x_n)$ is more concentrated
in specific regions of the latent space (the blue contours in the image).
However, by sampling from $q_{\theta}(z_n|x_n)$, we can sample from a distribution that is more likely to capture the structure of the data.
The variational distribution $q_{\theta}(z_n|x_n)$ is designed to be a good approximation of the true posterior $p(z_n|x_n)$,
Hence, samples drawn from $q_{\theta}(z_n|x_n)$ are more likely to be in regions of the latent space that contribute to the log-likelihood. 
In other words, the samples drawn from $q_{\theta}(z_n|x_n)$ are concentrated in areas where the posterior is high, which makes tgese
samples more useful. This lead to a more efficient Monte-Carlo integration because fewer samples are needed to estimate the
log-likelihood accurately.

\subsection*{1.7}

When we sample directly from a distribution $q_{\theta}(z_n|x_n)$ to estimate the expectation $\mathbb{E}_{q_{\theta}(z_n|x_n)}[\log p(x_n|z_n)]$,
the sampling part is non-differentiable with respect to the parameters $\theta$ of the model. This means that
the gradients with respect to the encoder parameters $\theta$ cannot be computed directly since the sampling operation does
not provide a smooth path for backpropagation. This is because the sampling operation introduces a discontinuity in the
computation graph, which makes it impossible to compute the gradients using standard backpropagation.

The reparameterization trick is a technique used to make the sampling operation differentiable with respect to the parameters $\theta$.
The trick involves reparameterizing the random variable $z_n$ as a deterministic function of the encoder outputs
$\mu_{\theta}(x_n)$ and $\sigma_{\theta}(x_n)$, and a noise variable $\epsilon$ sampled from a fixed distribution, such as a standard normal distribution,
i.e., $\epsilon \sim \mathcal{N}(0, I_D)$. This way the sample $z_n$ can be expressed as:

\begin{align*}
    z_n = \mu_{\theta}(x_n) + \sigma_{\theta}(x_n) \odot \epsilon \\
\end{align*}

With this, the sampling operation is now differentiable with respect to the parameters $\theta$ of the model. Allowing
us to compute the gradients of the loss function with respect to the parameters $\theta$ using backpropagation through the
encoder network. 

\subsection*{1.8}

\textcolor{red}{TODO}

\subsection*{1.9}

\textcolor{red}{TODO}

\subsection*{1.10}

\textcolor{red}{TODO}

\section*{Part 2}

\subsection*{2.1}

\textcolor{red}{TODO}

\subsection*{2.2}

\textcolor{red}{TODO}

\subsection*{2.3}

\textcolor{red}{TODO}



























\newpage
\bibliographystyle{abbrvnat}
\bibliography{references} 


\end{document}